

\documentclass{article}
\usepackage[utf8]{inputenc}
\usepackage{setspace}
\usepackage{ mathrsfs }
\usepackage{amssymb} %maths
\usepackage{amsmath} %maths
\usepackage[margin=0.2in]{geometry}
\usepackage{graphicx}
\usepackage{ulem}
\setlength{\parindent}{0pt}
\setlength{\parskip}{10pt}
\usepackage{hyperref}
\usepackage[autostyle]{csquotes}

\usepackage{cancel}
\renewcommand{\i}{\textit}
\renewcommand{\b}{\textbf}
\newcommand{\q}{\enquote}
\newcommand{\p}{$\phi \ $}

\renewcommand{\H}{\Bbb H}
%\renewcommand{\l}[1]{\lceil #1 \rceil }
\renewcommand{\l}[1]{ #1  }
%\newcommand{\lte}{\sqsubset}
%\newcommand{\lte}{\angle}
\newcommand{\lte}{\preceq}




%\vskip1.0in



\begin{document}
%\begin{huge}
{\setstretch{0.0}{

CANDLES 

\section{Definition}
A \b{candle} is a bounded-above set of \b{positive} rationals with no greatest element which is closed downward. A candle is like a Dedekind cut, but it's of finite length. The set of rational numbers involved should be visualized like a rod for measuring with. The candle metaphor isn't as direct a reference to measurement as the rod metaphor, but it \i{does} get the action and complexity at the \q{top} of the \q{candle} right. For this is where the real number strangeness lives. Our candles contain no greatest element. This isn't so exciting for rational candles, but for candles it takes us right into the heart of the real numbers. 

\section{Rational Candles}
The simplest candles, written in the most convenient and simplified notation, are of the form $(q < q_0)$, with $q_o \in \Bbb Q_{>0}$ such as $(q < 2)$. Just for clarity, these could be written more formally as $\{ q \in \Bbb Q : 0 < q < q_0\}$ and $\{ q \in \Bbb Q : 0 < q < 2\}$. We can get away with brief notation by remembering that our only ingredient is the strictly positive rational numbers. 

Since $\Bbb C$ is taken by the complex numbers, we'll let $\Bbb I$ symbolize the set of candles. Then our injection of the positive rationals into the candles is $f : \Bbb Q_{>0} \to \Bbb I $ defined by $f(p) = (q < p)$. 

\section{Order}
The order of the candles is provided by the subset relation. If $A \subsetneq B$ then $A < B$. We can also say that $A < B$ whenever $\exists q \in A - B$. 

\section{Addition}
Addition is as simple as one could hope, with $A + B := \{a + b : a \in A, b \in B \}$, which we can abbreviate as \{a+b\}. 


\section{Multiplication}
The multiplication of candles is just as easy. We define the product $AB$ as the standard set product, so that $AB := \{ab : a \in A, b \in B \}$, abbreviated as $\{ab\}$. Note that the simplicity of these basic operations is a motivation for constructing the candles and not the cuts.


\section{Simple Square Roots}
This simplified parenthetical set notation gets fun as we move on to square roots.  I will sketch a proof that the candle $f(2) = (q < 2)$ has a square root in the system, which is of course $( q^2 < 2)$. 

First let $x \in ( q^2 < 2)( q^2 < 2)$. Then $x < ab$ with $a^2 < q$ and $ b^2 < q$. So $x < \max(a,b)^2 < 2$. Since $\max(a,b)^2 \in (q < 2)$ and $(q < 2)$ is closed downward, $x$ is itself in $(q < 2)$. This gives us $( q^2 < 2)( q^2 < 2) = ( q^2 < 2)^2 \subset (q < 2).$

Now let $x \in (q < 2)$. More detail is available in the supplement, but we basically use analysis on the rational numbers to see this. Let Since $x < 2$, there's space for a $p$ such that $x < p^2 < 2$. Consider for instance the infinite decimal sequence expansion $p_n$ of $\sqrt{2}$. Each term is this sequence is rational, and the terms satisfy $p_n^2 < 2$ for all $n$, and $2 - p_n^2 \to 0$.  So we have some $p$ such that $x < p^2 < 2$. This puts $p$ in $( q^2 < 2)$ and so $x < p^2 \in ( q^2 < 2)^2$. Hence $x \in ( q^2 < 2)^2$. So  $(q < 2) \subset ( q^2 < 2)^2$, and the sets are equal. 

Of course it's not ideal to appeal to limits when building a system in which limits become possible in the first place. In the supplement files, I take a different, simpler approach. 

\section{Generalized Square Roots}

We can also define a \q{generalized} square root for candles that are not rational, for any candle whatsoever, using of course only the abstract properties of an indeterminate candle. 

Define $\sqrt{A} = \{ q^2 \in A \}$. Note that we move from parentheses to set brackets, just to emphasize the new abstraction. The proof is only a little more difficult than the one above. It's in the supplement, so I omit it here. 
 
\section{Multiplicative Inverse}

Define $A^{-c}$ to be the set inverse of the complement of $A \in \Bbb I$. In other words, let $A^{-c} = \{ 1/b : b \notin A\}$. Then $A^{-c}A = (q < 1),$ so that we have our inverse. The proof is made easier by noticing that we can always find a bounded increasing sequence $a_n < a_{n+1}$ such that $a_n + 2^{-n} \notin A$.  Details are available in the supplement files. 

\section{Limits of Increasing Bounded Sequences}

While a limited version of subtraction is possible, we can also define a nontraditional \q{limit} of any bounded-above increasing sequence of candles. Let $A_n \leq A_{n+1} < B$. Then $A_{\infty}   = \lim A_n := \cup_{n =1}^{\infty} A_n$. It's not hard to prove that this is a candle in general, and it then serves as a shortcut for defining the equivalents of numbers like $e$ in the candles. 


\section{Variants}
Here I've used all the positive rationals, but it's also fun to work with dyadics or other dense subsets, and some of the supplementary files take this other approach.


\section{Motivation}
 Of course there's nothing \i{mathematically} revolutionary here. But I value the candle system (and the ladder system) pedagogically and aesthetically as intuition pumps. I like constructions that make the real numbers seem a little less \b{unreal}, and yet I want to make their strange properties manifest to students and others who take them for granted
 
I present these simplified \q{realish} number systems (the candles above are the simplest) as more accessible cousins of the Dedekind cuts. The Dedekind cuts are brilliant, but the details can be overwhelming. Can we live without negatives and subtraction when the point is to fill the gaps in the rational numbers ? Or rather to show \i{how} a more complex and adequate number system can be created from scratch from a simpler and more familiar number system. As a math teacher, I'd be most delighted by a students decision to see if they could make their own system, while taking the trouble to make sure it actually works. 


%\end{huge}
\end{document}
