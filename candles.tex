

\documentclass{article}
\usepackage[utf8]{inputenc}
\usepackage{setspace}
\usepackage{ mathrsfs }
\usepackage{amssymb} %maths
\usepackage{amsmath} %maths
\usepackage[margin=0.2in]{geometry}
\usepackage{graphicx}
\usepackage{ulem}
\setlength{\parindent}{0pt}
\setlength{\parskip}{10pt}
\usepackage{hyperref}
\usepackage[autostyle]{csquotes}

\usepackage{cancel}
\renewcommand{\i}{\textit}
\renewcommand{\b}{\textbf}
\newcommand{\q}{\enquote}
\newcommand{\p}{$\phi \ $}

\renewcommand{\H}{\Bbb H}
%\renewcommand{\l}[1]{\lceil #1 \rceil }
\renewcommand{\l}[1]{ #1  }
%\newcommand{\lte}{\sqsubset}
%\newcommand{\lte}{\angle}
\newcommand{\lte}{\preceq}




%\vskip1.0in



\begin{document}
%\begin{huge}
{\setstretch{0.0}{

CANDLES 

\section{Definition}
A \b{candle} is a bounded-above set of \b{positive} rationals with no greatest element which is closed downward. A candle is like a Dedekind cut, but it's of finite length. The set of rational numbers involved should be visualized like a rod, as something to measure with.  The candle metaphor isn't as direct a reference to the rod metaphor, but it \i{does} get the action and complexity at the top of the \q{candle} (the top of the set of rationals) right. 

\section{Rational Candles}
The simplest candles, written in the most convenient and simplified notation, are of the form $(q < q_0)$, such as $(q < 2)$. Just for clarity, these could be written more formally as $\{ q \in \Bbb Q : 0 < q < q_0\}$ and $\{ q \in \Bbb Q : 0 < q < 2\}$. We can't get away with brief notation by remembering that our only ingredient is the strictly positive rational numbers. 

Note that in  the form $(q < q_0)$ that $q_0$ is supposed to symbolize a rational number, and this symbol form is also how we inject positive rationals into the candles. Since $\Bbb C$ is taken by the complex numbers, we'll let $\Bbb I$ symbolize the set of candles. Then our inject is $f : \Bbb Q_{>0} \to \Bbb I $ defined by $f(p) = (q < p)$ 

\section{Simple Square Roots}
This simplified parenthetical set notation gets fun as we move on to square roots. Once we define muliplication, it's not so hard to prove that $(q^2 < 2)(q^2 < 2) = (q < 2)$


The most natural way to define the product $AB$ of candles $A$ and $B$ would just be $\{ab : a \in A, b \in B \}$. But various proofs are easier if we build a slight more complex product. So we define $AB = \{ q \in Q : \ \ \exists a \in A \quad \exists b \in B \quad 0 < q < ab \}.$ That this product is itself a candle is proved in the supplementing documents. 

But I will give a quick proof that the candle $f(2) = (q < 2)$ has a square root in the system, which is of course $( q^2 < 2)$. 

First let $x \in ( q^2 < 2)( q^2 < 2)$. Then $x < ab$ with $a^2 < q$ and $ b^2 < q$. So $x < \max(a,b)^2 < 2$. Since $\max(a,b)^2 \in (q < 2)$ and $(q < 2)$ is closed downward, $x$ is itself in $(q < 2)$. This gives us $( q^2 < 2)( q^2 < 2) = ( q^2 < 2)^2 \subset (q < 2).$

Now let $x \in (q < 2)$. More detail is available in the supplement, but we basically use analysis on the rational numbers to see this. Let Since $x < 2$, there's space for a $p$ such that $x < p^2 < 2$. Consider for instance the infinite decimal sequence expansion $p_n$ of $\sqrt{2}$. Each term is this sequence is rational, and the terms satisfy $p_n^2 < 2$ for all $n$, and $2 - p_n^2 \to 0$.  So we have some $p$ such that $x < p^2 < 2$. This puts $p$ in $( q^2 < 2)$ and so $x < p^2 \in ( q^2 < 2)^2$. Hence $x \in ( q^2 < 2)^2$. So  $(q < 2) \subset ( q^2 < 2)^2$, and the sets are equal. 

Of course it's not ideal to appeal to limits when building a system in which limits become possible in the first place. In the supplement files, I take a different, simpler approach. 

\section{Generalized Square Roots}

We can also define a \q{generalized} square root for candles that are not rational, for any candle whatsoever, using of course only the abstract properties of an indeterminate candle. 

Define $\sqrt{A} = \{ q^2 \in A \}$. Note that we move from parentheses to set brackets, just to emphasize the new abstraction. The proof is only a little more difficult than the one above. It's in the supplement, so I omit it here. 
 

\section{Variants}
Here I've used all the positive rationals, but it's also fun to work with dyadics or other dense subsets, and some of the supplementary files take this other approach.


\section{Motivation}
 Of course there's nothing mathematically revolutionary here. I value the candle system (and the ladder system) as intuition pumps. I like constructions that make the real numbers seem a little less \b{unreal}, and yet I want to make their strange properties manifest in their strangeness. So I think in terms of simplified \i{realish} systems that might be taught as recreational or just mind-opening mathematics to nonmathematicians. I teach nonmathematics majors and I have the sense that they are missing out largely on the joy and depth and creativity of math. So I reach for the vivid metaphor, and I value any simplifications that make the main point easier to grasp. Here of course I try to give the easiest path to \b{creating} a square root for $2$, a number that we tend to take for granted until we take a serious look at the real numbers. But we don't even need to define addition and negative numbers to create an entity in an analogous system. And that's what the candles might be good for. 



%\end{huge}
\end{document}
